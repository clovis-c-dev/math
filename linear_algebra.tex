\documentclass{article}
\usepackage{amsmath, amssymb, amsthm}
\usepackage{enumitem}
\usepackage{xcolor}  % For colored text

% Define theorem styles without mdframed
\theoremstyle{plain}
\newtheorem{theorem}{Theorem}[section]
\newtheorem{lemma}[theorem]{Lemma}
\newtheorem{proposition}[theorem]{Proposition}

\theoremstyle{definition}
\newtheorem{definition}{Definition}[section]
\newtheorem{example}{Example}[section]

\theoremstyle{remark}
\newtheorem{remark}{Remark}[section]

% Custom colors for theorems
\definecolor{theoremcolor}{RGB}{0,0,128}
\definecolor{defcolor}{RGB}{0,100,0}
\definecolor{excolor}{RGB}{139,69,19}

% Custom theorem headers
\makeatletter
\newcommand{\thmheader}[1]{%
  \textcolor{#1}{\normalfont\bfseries}%
}
\makeatother

\title{Vector Spaces in Linear Algebra}
\author{Clovis W. Bertholini Sb.}
\date{\today}

\begin{document}

\maketitle

\section{Vector Space Axioms}
\begin{definition}
\thmheader{defcolor}A \textbf{vector space} over a field $\mathbb{F}$ is a set $V$ with two operations:
\begin{itemize}[leftmargin=*]
    \item \textbf{Vector addition}: $+: V \times V \to V$
    \item \textbf{Scalar multiplication}: $\cdot: \mathbb{F} \times V \to V$
\end{itemize}
satisfying these axioms:
\end{definition}

\begin{enumerate}[label=(V\arabic*),leftmargin=*]
    \item $\mathbf{u} + \mathbf{v} = \mathbf{v} + \mathbf{u}$
    \item $(\mathbf{u} + \mathbf{v}) + \mathbf{w} = \mathbf{u} + (\mathbf{v} + \mathbf{w})$
    \item $\exists \mathbf{0} \in V$ such that $\mathbf{v} + \mathbf{0} = \mathbf{v}$
    \item $\forall \mathbf{v} \in V, \exists -\mathbf{v} \in V$ with $\mathbf{v} + (-\mathbf{v}) = \mathbf{0}$
    \item $\alpha(\beta\mathbf{v}) = (\alpha\beta)\mathbf{v}$
    \item $1\mathbf{v} = \mathbf{v}$
    \item $\alpha(\mathbf{u} + \mathbf{v}) = \alpha\mathbf{u} + \alpha\mathbf{v}$
    \item $(\alpha + \beta)\mathbf{v} = \alpha\mathbf{v} + \beta\mathbf{v}$
\end{enumerate}

\section{Important Properties}
\begin{theorem}
\thmheader{theoremcolor}In any vector space $V$:
\begin{itemize}[leftmargin=*]
    \item The zero vector $\mathbf{0}$ is unique
    \item Additive inverses are unique
    \item $0\mathbf{v} = \mathbf{0}$ for all $\mathbf{v} \in V$
    \item $(-1)\mathbf{v} = -\mathbf{v}$ for all $\mathbf{v} \in V$
\end{itemize}
\end{theorem}

\section{Key Examples}
\begin{example}
\thmheader{excolor}\textbf{Euclidean Space $\mathbb{R}^n$}:
\[
\mathbb{R}^n = \left\{ \begin{pmatrix} x_1 \\ \vdots \\ x_n \end{pmatrix} \mid x_i \in \mathbb{R} \right\}
\]
with component-wise operations.
\end{example}

\begin{example}
\thmheader{excolor}\textbf{Matrix Space $M_{m \times n}(\mathbb{R})$}:
\[
\begin{pmatrix} a & b \\ c & d \end{pmatrix} + \begin{pmatrix} e & f \\ g & h \end{pmatrix} = \begin{pmatrix} a+e & b+f \\ c+g & d+h \end{pmatrix}
\]
\end{example}

\end{document}
